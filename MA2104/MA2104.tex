\documentclass[11pt]{article}

%:Packages
\usepackage{geometry}                	                  
\usepackage{amssymb}
\usepackage{amsmath}
\usepackage{amsfonts} 	  % \mathbb commands
\usepackage{enumitem} 	  % control description indentation
\usepackage{tcolorbox}
\usepackage{mathtools}

%:Cover Page
\begin{document}
\title{Notes for MA2104, AY24/25 Sem 2}
\author{Michael Yang}
\date{9 Jan 2025}
\maketitle

%:Custom Commands

\newcommand{\R}[1]{\mbox{$\mathbb{R}^{#1}$}}  % R^n
\newcommand{\Bf}[1]{\mathbf{#1}}
\newcommand{\Bb}[1]{\mathbb{#1}}
\newcommand{\q}{\quad}
\newcommand{\qq}{\qquad}
\newcommand{\br}{\\\;\\}

\newcommand{\sectiontitle}[1]{% Section title
\begin{flushleft}\large{\textbf{#1}}\end{flushleft}} 

\newcommand{\chapter}[1]{% New chapter
\newpage
\section*{#1}
\hrule
\vspace{0.3cm}
}

\newcommand{\vv}[1]{%
\langle 
{#1}
\rangle
}

%:Margins
\newgeometry{left=1cm, right=1cm, top=1cm, bottom=2cm} 

\chapter{3-Dimensional Space}

\sectiontitle{Equations of Lines}

Vector Form of the equation of a line

\[ \vec{r}=\vec{r_{0}}+t\vec{v}=\vv{x_{0},y_{0},z_{0}}+t\vv{a,b,c} \] 

\[ \vv{x,y,z}=\vv{x_{0}+ta,y_{0}+tb,z_{0}+tc} \] \\

Parametric Form of the equation of a line

\begin{align*}
x&=x_{0}+ta \\
y&=y_{0}+tb \\
z&=z_{0}+tc \\
\end{align*}

Symmetric Equations of the line

\[ \frac{x-x_{0}}{a}=\frac{y-y_{0}}{b}=\frac{z-z_{0}}{c} \]

\sectiontitle{Equations of Planes}

Vector Equation of Plane

\[ \vec{n} \cdot (\vec{r}-\vec{r_{0}}) = 0 \qq \Rightarrow \qq \vec{n} \cdot \vec{r} = \vec{n}\cdot \vec{r_{0}} \]

\[ \vv{a,b,c}\cdot (\vv{x,y,z}-\vv{x_{0},y_{0},z_{0}}) = 0 \] 

\[ \vv{a,b,c}\cdot\vv{x-x_{0},y-y_{0},z-z_{0}} = 0 \] \\

Scalar Equation of Plane

\[ a(x-x_{0})+b(y-y_{0})+c(z-z_{0})=0 \]

\[ ax+by+cz=d\]

where $d=ax_{0}+by_{0}+cz_{0}$ \\

If $\vec{a}$ and $\vec{b}$ are parallel: $\vec{a}\times\vec{b}=\vec{0}$ \\

If $\vec{a}$ and $\vec{b}$ are orthogonal: $\vec{a}\cdot\vec{b}=0$ \\

\sectiontitle{Quadric Surfaces}

Quadric surfaces are the graphs of any equation that can be put into the general form

\[ Ax^{2}+By^{2}+Cz^{2}+Dxy+Exz+Fyz+Gx+Hy+Iz+J=0 \]

where $A$ to $J$ are constants \\

Ellipsoid

\[\frac{x^{2}}{a^{2}}+\frac{y^{2}}{b^{2}}+\frac{z^{2}}{c^{2}} = 1\] 

If $a=b=c$, it is a sphere \\

Cone / Hourglass

\[ \frac{x^{2}}{a^{2}} + \frac{y^{2}}{b^{2}} = \frac{z^{2}}{c^{2}} \qq \text{(Open up on the z-axis)} \]
\[ \frac{y^{2}}{b^{2}} + \frac{z^{2}}{c^{2}} = \frac{x^{2}}{a^{2}} \qq \text{(Open up on the x-axis)} \]
\[ \frac{x^{2}}{a^{2}} + \frac{z^{2}}{c^{2}} = \frac{y^{2}}{b^{2}} \qq \text{(Open up on the y-axis)} \] \\

Cylinder

\[ \frac{x^{2}}{a^{2}}+\frac{y^{2}}{b^{2}}=1\]

If $a=b$ we have a cylinder whose cross section is a circle

\[ x^{2}+y^{2}=r^{2}\]

The cylinder will be centered on the axis corresponding to the variable that does not appear in the equation. 

In two dimensions it is a circle, but in three dimensions it is a cylinder.\\

Hyperboloid of One Sheet

\[ \frac{x^{2}}{a^{2}} + \frac{y^{2}}{b^{2}} - \frac{z^{2}}{c^{2}} = 1\]

The variable with the negative in front of it will give the axis along which the graph is centered.\\

Hyperboloid of Two Sheets

\[ -\frac{x^{2}}{a^{2}} - \frac{y^{2}}{b^{2}} + \frac{z^{2}}{c^{2}} = 1 \]

The variable with the positive in front of it will give the axis along which the graph is centered. 

The only difference between the hyperboloid of one sheet and the hyperboloid of two sheets is the signs.

Elliptic Paraboloid

\[ \frac{x^{2}}{a^{2}} + \frac{y^{2}}{b^{2}} = \frac{z}{c} \]

The variable that isn’t squared determines the axis upon which the paraboloid opens up. 

The sign of $c$ will determine the direction that the paraboloid opens.

If $c$ is positive then it opens up and if $c$ is negative then it opens down.

Hyperbolic Paraboloid

\[ \frac{x^{2}}{a^{2}} - \frac{y^{2}}{b^{2}} = \frac{z}{c} \]

For both types of paraboloids, the surface can be easily moved up or down by adding/subtracting a constant.\\

\sectiontitle{Multivariable Functions}

\end{document}
